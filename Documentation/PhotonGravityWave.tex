\documentclass{article}
\usepackage{amsmath}
\begin{document}
\begin{flushleft}
{\large Building a Numerical Model of the Influence of a Gravitational Wave on an Electromagnetic Wave}

The Physics behind PhotonGravityWavePerpendicularCase.py and PhotonGravityWave4D.py

\smallskip
Varadarajan Srinivasan
\bigskip
\end{flushleft}

Let us define a left-handed coordinate system along whose xy plane electromagnetic waves can travel. Consider a gravitational wave propagating along the z-axis. Let us further define the x- and y-axes to be along the polarization axes of the gravitational wave. This has the effect of stretching and contracting space along those 2 axes. So, the wave equation in the presence of a gravitational wave can be written (taking out the the time component because it will not be affected) as
\begin{equation} \label{waveeqn with grav}
\begin{pmatrix}\partial_x & \partial_y & \partial_z\end{pmatrix} 
\textbf{M}
\begin{pmatrix} \partial_x \\ \partial_y \\ \partial_z \end{pmatrix}
u
=\frac{1}{c^2}\frac{\partial^2 u}{\partial t^2}
\end{equation}
where $\textbf{M}$ is the tensor represented by
\begin{equation} \label{M}
\textbf{M}=\begin{pmatrix}
1+f \\
 & 1-f \\
 & & 1
\end{pmatrix}
\text{ \ \ \ where } f=\epsilon \cos(kz-kt)
\end{equation}
Note that when the amplitude $\epsilon$ is 0, Eq. \ref{waveeqn with grav} reduces to the standard wave equation for flat spacetime,
\begin{equation} \label{waveeqn flat}
\nabla^2u=\frac{1}{c^2}\frac{\partial^2u}{\partial t^2}.
\end{equation}

Let us generalize this for any inclination of the gravitational wave. We need not generalize for azimuthal angle because we can always define a planar axis along the azimuthal direction of the wave. First, let us define a left-handed coordinate system attached to the gravitational wave with $x\prime,y\prime$ along its polarization axes of and $z\prime$ as its propagation direction. Second, we must also define a left-handed coordinate system $\langle X,Y,Z \rangle$ fixed to our frame of interest whose XY plane contains those electromagnetic waves. The latter is the frame our computational algorithms will use; the rows and columns of the numerical method form the discretized representation of the plane. The row indices and column indices correspond to Y-values and X-values respectively with (0,0) as the top left corner of the left-handed system. Note that our initial coordinate system $\langle x,y,z\rangle$ was equivalent to both $\langle X,Y,Z\rangle$ and $\langle x\prime,y\prime,z\prime\rangle$ as that was the case where the gravitational wave is perpendicular to the plane. In our new frame, our tensor $\textbf{M}$ takes $f=\epsilon \cos(kz\prime-kt)$ with t unchanged.

We know that we can always find a rotation matrix $\textbf{R}$ such that
\begin{equation} \label{R}
\begin{pmatrix}
X \\
Y \\
Z
\end{pmatrix}=\textbf{R}\begin{pmatrix}
x\prime \\
y\prime \\
z\prime
\end{pmatrix}
\end{equation}. We can now rewrite the left hand side of Eq. \ref{waveeqn with grav} as
\begin{align}
LHS &= \begin{pmatrix}\partial_{x\prime} & \partial_{y\prime} & \partial_{z\prime}\end{pmatrix} 
\textbf{R}^{-1}\textbf{R} \  \textbf{M} \ \textbf{R}^{-1}\textbf{R}
\begin{pmatrix} \partial_{x\prime} \\ \partial_{y\prime} \\ \partial_{z\prime} \end{pmatrix}
u
\\ &= \begin{pmatrix}\partial_{x\prime} & \partial_{y\prime} & \partial_{z\prime}\end{pmatrix} 
\textbf{R}^{-1}\textbf{R} \  \textbf{M} \ \textbf{R}^{-1}
\begin{pmatrix} \partial_X \\ \partial_Y \\ \partial_Z \end{pmatrix}
u
\end{align}



\end{document}