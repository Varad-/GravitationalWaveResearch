\documentclass{article}
\usepackage{amsmath}
\begin{document}
\begin{flushleft}
{\large Building a Numerical Model of the Influence of a Gravitational Wave on an Electromagnetic Wave}

The Physics behind PhotonGravityWavePerpendicularCase.py and PhotonGravityWave4D.py

\smallskip
Varadarajan Srinivasan
\smallskip
\end{flushleft}

Let us define a left-handed\footnote{This orientation is more convenient for our computational system because its natural 2D origin is at the top-left corner.} coordinate system along whose xy plane electromagnetic waves can travel. Consider a gravitational wave propagating along the z-axis. Let us further define the x- and y-axes to be along the polarization axes of the gravitational wave. This has the effect of stretching and contracting space along those 2 axes. So, the wave equation in the presence of a gravitational wave can be written (taking out the the time component because it will not be affected) as
\begin{equation} \label{waveeqn with grav}
\begin{pmatrix}\partial_x & \partial_y & \partial_z\end{pmatrix} 
\textbf{M}
\begin{pmatrix} \partial_x \\ \partial_y \\ \partial_z \end{pmatrix}
u
=\frac{1}{c^2}\frac{\partial^2 u}{\partial t^2},
\end{equation}
where $\textbf{M}$ is the space component of the metric tensor and is represented by
\begin{equation} \label{M}
\textbf{M}=\begin{pmatrix}
1+f \\
 & 1-f \\
 & & 1
\end{pmatrix},
\text{ \ \ \ where } f=\epsilon \cos(kz-kct).
\end{equation}
Note that when the amplitude $\epsilon$ is 0, Eq. \ref{waveeqn with grav} reduces to the standard wave equation with no gravitational wave (i.e. flat spacetime),
\begin{equation} \label{waveeqn flat}
\nabla^2u=\frac{1}{c^2}\frac{\partial^2u}{\partial t^2}.
\end{equation}

Let us generalize this for any inclination of the gravitational wave. We need not generalize for azimuthal angle because we can always define a planar axis along the azimuthal direction of the wave. First, let us define a left-handed coordinate system $\langle x\prime,y\prime,z\prime\rangle$ attached to the gravitational wave such that $x\prime$ and $y\prime$ are along its polarization axes with $z\prime$ along its propagation direction. Second, we must also define a left-handed coordinate system $\langle X,Y,Z \rangle$ fixed to our frame of interest whose XY plane contains those electromagnetic waves. The latter is the frame our computational algorithms will use; the rows and columns of the numerical method form the discretized representation of the plane. The row indices and column indices correspond to Y-values and X-values respectively with (0,0) as the top-left corner. Note that our initial coordinate system $\langle x,y,z\rangle$ was equivalent to both $\langle X,Y,Z\rangle$ and $\langle x\prime,y\prime,z\prime\rangle$ as that was the case where the gravitational wave is perpendicular to the plane. In our new frame, our tensor $\textbf{M}$ takes $f=\epsilon \cos(kz\prime-kct)$ with t unchanged.

We know that we can always find a rotation matrix $\textbf{R}$ such that
\begin{equation} \label{R}
\begin{pmatrix}
X \\
Y \\
Z
\end{pmatrix}=\textbf{R}\begin{pmatrix}
x\prime \\
y\prime \\
z\prime
\end{pmatrix}.
\end{equation} Since the rotation matrix must be orthogonal, $\textbf{R}^{-1}=\textbf{R}^T$. Therefore, transposing both sides of Eq. \ref{R} shows that $\begin{pmatrix} X & Y & Z\end{pmatrix} = \begin{pmatrix}x\prime & y\prime & z\prime\end{pmatrix}\textbf{R}^{-1}.$ We can now rewrite the left hand side of Eq. \ref{waveeqn with grav} as
\begin{align}
LHS &= \begin{pmatrix}\partial_{x\prime} & \partial_{y\prime} & \partial_{z\prime}\end{pmatrix} 
\textbf{R}^{-1}\textbf{R} \  \textbf{M} \ \textbf{R}^{-1}\textbf{R}
\begin{pmatrix} \partial_{x\prime} \\ \partial_{y\prime} \\ \partial_{z\prime} \end{pmatrix}
u \nonumber
\\ &= \begin{pmatrix}\partial_X & \partial_Y & \partial_Z\end{pmatrix}
\textbf{R} \  \textbf{M} \ \textbf{R}^{-1}
\begin{pmatrix} \partial_X \\ \partial_Y \\ \partial_Z \end{pmatrix}
u. \label{LHS}
\end{align}
Let us define another tensor $\textbf{T}\equiv\textbf{RMR}^{-1}$, the conjugate of $\textbf{M}$ by rotation. We can always define $\langle X,Y,Z\rangle$ with respect to the incoming gravitational wave such that the only rotation we need to do is about $x\prime$.\footnote{We are actually rotating about $z\prime$, then $x\prime$, and finally about the new $z\prime$, but the roll rotations do not change anything. So, those Euler angles do not enter our process.} Thus, for an angle of incidence of $\theta$,\footnote{Note that $\theta$ is defined clockwise when viewed from above the YZ plane because our coordinate system is left-handed.}
\begin{equation*}
\textbf{T}=
\begin{pmatrix}
1 & 0 & 0 \\
0 & \cos(\theta) & -\sin(\theta) \\
0 & \sin(\theta) & \cos(\theta)
\end{pmatrix}
\begin{pmatrix}
1+f & 0 & 0 \\
0 & 1-f & 0 \\
0 & 0 & 1
\end{pmatrix}
\begin{pmatrix}
1 & 0 & 0 \\
0 & \cos(\theta) & \sin(\theta) \\
0 & -\sin(\theta) & \cos(\theta)
\end{pmatrix}.
\end{equation*}
We introduce the parameter $\theta$ to convert between our two coordinate systems as our computations require all terms to be in the $\langle X,Y,Z\rangle$ system, but recall that $f=f(t,z\prime)=\epsilon \cos(kz\prime-kct)$. Solving Eq. \ref{R} for the primed column vector with $\textbf{R}$ as above about $x\prime$,
\begin{equation*}
\begin{pmatrix}
x\prime \\
y\prime \\
z\prime
\end{pmatrix}
= \textbf{R}^{T}\begin{pmatrix}
X \\
Y \\
Z
\end{pmatrix}
=\begin{pmatrix}
X \\
Y\cos(\theta)+Z\sin(\theta) \\
-Y\sin(\theta)+Z\
\end{pmatrix}.
\end{equation*}
Thus, $f=f(t,Y,Z)=\epsilon \cos(k(Z-Y\sin(\theta))-kct)$. This gives us the desired compact form of Eq. \ref{LHS} entirely in our computational coordinate system $\langle X,Y,Z\rangle$,
\begin{equation*}
LHS = \nabla^T \textbf{T} \nabla u.
\end{equation*}
Hence, we have the (3+1)-dimensional equation for displacement $u(t,X,Y,Z)$ of an electromagnetic field obeying the Wave Equation while interacting with a gravitational wave,
\begin{equation}\label{main eqn}
\boxed{
\nabla^T \textbf{T} \nabla u = \frac{1}{c^2}\frac{\partial^2 u}{\partial t^2} \ \ \ \ \mbox{with } f=f(t,Y,Z).
}
\end{equation}

To numerical compute results, we must discretize Eq. \ref{main eqn}. Our process for this will build on that already described in WaveEquation.pdf. Our plane of interest containing the electromagnetic waves is the XY plane and so we need only concern ourselves with the case $f=f(t,Y,0)=\epsilon \cos(kY\sin(\theta)+kct).$
\end{document}