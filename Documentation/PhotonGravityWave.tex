\documentclass{article}
\usepackage{amsmath,scalefnt,fullpage,fancyvrb}

\begin{document}
\begin{flushleft}
{\large Building a Numerical Model of the Influence of a Gravitational Wave on an Electromagnetic Four-Potential}
\end{flushleft}
The Physics behind
\vspace{-2ex}
\begin{verbatim}
PhotonGravityWavePerpendicularCase(2+1)-D.py,
PhotonGravityWaveInefficient(2+1)-D.py,
PhotonGravityWave(2+1)-D.py.
\end{verbatim}
\begin{flushright}
\bigskip
Varadarajan Srinivasan
\end{flushright}
\setlength{\rightskip}{1cm}
\setlength{\leftskip}{1cm}
Note: All of the following is done in full 3+1 dimensions, but the implementation in Python is in (2+1)-D because that makes the resulting animated visualizations less cumbersome to display, greatly increases the resolution-to-runtime ratio, and suffices to show the desired effects. In fact, some subtler effects are only noticeable in (2+1)-D. Nevertheless, the program is structured in such a way that extending it to (3+1)-D would be quick and straightforward.

\setlength{\leftskip}{0pt}
\setlength{\rightskip}{0pt}

\bigskip

Let us define a left-handed\footnote{This orientation is more convenient for our computational system because its natural 2D origin is at the top-left corner.} coordinate system in the presence of an electromagnetic field. Consider a gravitational wave propagating along the z-axis. Let us define further the x- and y-axes to be along the polarization axes (along which space stretches and contracts) of the gravitational wave. This affects Maxwell's Equations such that the wave equation for a component $u$ of the electromagnetic four-potential in the influence of a gravitational wave can be written (separating the unaffected time component) as
\begin{equation} \label{waveeqn with grav}
\begin{pmatrix}\partial_x & \partial_y & \partial_z\end{pmatrix} 
\textbf{g}_S
\begin{pmatrix} \partial_x \\ \partial_y \\ \partial_z \end{pmatrix}
u
=\frac{1}{c^2}\frac{\partial^2 u}{\partial t^2},
\end{equation}
where $\textbf{g}_S$ is the space component of the metric tensor and is represented by
\begin{equation} \label{M}
\textbf{g}_S=\begin{pmatrix}
1+f & 0 & 0 \\
0 & 1-f & 0 \\
0 & 0 & 1
\end{pmatrix},
\text{ \ \ \ where } f=\epsilon \cos(kz-kct).
\end{equation}
Note that when the amplitude $\epsilon$ is 0, Eq. \ref{waveeqn with grav} reduces to the standard wave equation with no gravitational wave (i.e. flat spacetime),
\begin{equation} \label{waveeqn flat}
\nabla^2u=\frac{1}{c^2}\frac{\partial^2u}{\partial t^2}.
\end{equation}

Let us generalize this for any inclination of the gravitational wave. We need not generalize for azimuthal angle because we can always define a planar axis along the azimuthal direction of the wave. First, let us define a left-handed coordinate system $\langle x',y',z'\rangle$ such that the polarization axes of the gravitational wave are parallel to the $x'$ and $y'$ axes and it propagates with velocity $c \hat z'$. Second, we must also define a left-handed coordinate system $\langle X,Y,Z \rangle$ fixed to our frame of computation. These frames are stationary with respect to each other; they differ only in their orientations. $\langle x',y',z'\rangle$ is oriented according to the gravitational wave and $\langle X,Y,Z \rangle$ is the fixed frame to which we will apply our computational algorithms. The row, column, depth, and time steps in our numerical method form a discrete representation of $\langle t,X,Y,Z \rangle$. Note that our initial coordinate system $\langle x,y,z\rangle$ was equivalent to both $\langle X,Y,Z\rangle$ and $\langle x',y',z'\rangle$ as we were only concerned with the case of the gravitational wave being perpendicular to the $XY$ plane. In our new frame, our tensor $\textbf{g}_S$ takes $f=\epsilon \cos(kz'-kct)$ with $t$ unchanged.

We know that we can always find a rotation matrix $\textbf{R}$ such that
\begin{equation} \label{R}
\begin{pmatrix}
X \\
Y \\
Z
\end{pmatrix}=\textbf{R}\begin{pmatrix}
x' \\
y' \\
z'
\end{pmatrix}.
\end{equation} Since the rotation matrix must be orthogonal, $\textbf{R}^{-1}=\textbf{R}^T$. Therefore, transposing both sides of Eq. \ref{R} shows that $\begin{pmatrix} X & Y & Z\end{pmatrix} = \begin{pmatrix}x' & y' & z'\end{pmatrix}\textbf{R}^{-1}.$ We can now rewrite the left hand side of Eq. \ref{waveeqn with grav} as
\begin{align}
LHS &= \begin{pmatrix}\partial_{x'} & \partial_{y'} & \partial_{z'}\end{pmatrix} 
\textbf{R}^{-1}\textbf{R} \  \textbf{g}_S \ \textbf{R}^{-1}\textbf{R}
\begin{pmatrix} \partial_{x'} \\ \partial_{y'} \\ \partial_{z'} \end{pmatrix}
u \nonumber \\
&= \begin{pmatrix}\partial_X & \partial_Y & \partial_Z\end{pmatrix}
\textbf{R} \  \textbf{g}_S \ \textbf{R}^{-1}
\begin{pmatrix} \partial_X \\ \partial_Y \\ \partial_Z \end{pmatrix}
u. \label{LHS}
\end{align}
Let us define another tensor $\textbf{T}\equiv\textbf{Rg}_S\textbf{R}^{-1}$, the conjugate of $\textbf{g}_S$ by rotation. We can characterize the rotation required to go from the gravitational wave coordinate system to our computational coordinate system as Euler angle rotations about $z',x',z'$, but we have enough freedom in our coordinates to be able to always orient our $\langle X,Y,Z\rangle$ such that the last roll rotation about the new $z'$ axis does nothing. Thus, for an angle of incidence of $\theta$,\footnote{Note that $\theta$ is defined clockwise when viewed from above the YZ plane because our coordinate system is left-handed.}
\begin{equation} \label{R explicit}
\textbf{R}=
\textbf{R}_{z'}\textbf{R}_{x'}
=
\left(
\begin{array}{ccc}
 \cos (\theta ) & -\cos (\theta ) \sin (\theta ) & \sin ^2(\theta ) \\
 \sin (\theta ) & \cos ^2(\theta ) & -\cos (\theta ) \sin (\theta ) \\
 0 & \sin (\theta ) & \cos (\theta ) \\
\end{array}
\right)
\end{equation}
and hence
\begin{equation} \label{T}
\textbf{T}=
\begin{pmatrix}
 \frac{1}{8} (4 f \cos (2 \theta )+f \cos (4 \theta )+3 f+8) & \frac{1}{8} f (6 \sin (2 \theta )+\sin (4 \theta ))
   & f \sin ^2(\theta ) \cos (\theta ) \\
 \frac{1}{8} f (6 \sin (2 \theta )+\sin (4 \theta )) & \frac{1}{8} (-8 f \cos (2 \theta )-f \cos (4 \theta )+f+8)
   & -f \sin (\theta ) \cos ^2(\theta ) \\
 f \sin ^2(\theta ) \cos (\theta ) & -f \sin (\theta ) \cos ^2(\theta ) & \frac{1}{2} (f \cos (2 \theta )-f+2) \\
\end{pmatrix}.
\end{equation}
We have introduced the parameter $\theta$ to convert between our two coordinate systems as our computations require all terms to be in the $\langle X,Y,Z\rangle$ system, but recall that $f=f(t,z')=\epsilon \cos(kz'-kct)$. We therefore need to solve Eq. \ref{R} for the primed column vector with $\textbf{R}$ given by Eq. \ref{R explicit}.
\begin{equation*}
\begin{pmatrix}
x' \\
y' \\
z'
\end{pmatrix}
= \textbf{R}^{T}\begin{pmatrix}
X \\
Y \\
Z
\end{pmatrix}
=
\left(
\begin{array}{c}
 X \cos (\theta )+Y \sin (\theta ) \\
 \frac{1}{2} (-X \sin (2 \theta )+Y \cos (2 \theta )+2 Z \sin (\theta )+Y) \\
 X \sin ^2(\theta )+\cos (\theta ) (Z-Y \sin (\theta )) \\
\end{array}
\right)
\end{equation*}
Thus, \begin{equation} \label{f}
f=f(t,X,Y,Z)=\epsilon \cos\big(k\big(X \sin ^2(\theta )+\cos (\theta ) \big(Z-Y \sin (\theta )\big)-ct\big)\big).
\end{equation}
This puts $LHS = \nabla^T \textbf{T} \nabla u$, the compact form of Eq. \ref{LHS}, entirely in terms of our computational coordinates $t,X,Y,Z$. Hence, we have the desired form of the generalized (3+1)-dimensional equation describing $u(t,X,Y,Z)$, a component of the electromagnetic four-potential interacting with a gravitational wave.
\begin{align}\label{main eqn}
\boxed{
\nabla^T \textbf{T} \nabla u = \frac{1}{c^2}\frac{\partial^2 u}{\partial t^2} \ \ \ \ \mbox{with } f=f(t,X,Y,Z)
}\\
\text{where $f(t,X,Y,Z)$ is given by Eq. \ref{f},} \nonumber \\
\text{and $\textbf{T}$ is given by Eq. \ref{T}.} \nonumber
\end{align}

To numerically compute results, we must discretize Eq. \ref{main eqn}. Our process for this will build on that described in WaveEquation.pdf. We define a numerical function\footnote{If we program in only 2+1 dimensions, we would ignore the third spatial index and Z-derivatives in all of these equations. This would also mean that $f=f(t,X,Y,0)=\epsilon \cos\big(k\big(X \sin ^2(\theta )-Y\cos (\theta ) \sin (\theta )-ct\big)\big)$.} $u_{m,n,o,t}$ where $m,n,o,$ and $t$ are the row, column, depth, and time indices, respectively. This means that, in our graphically conveniently left-handed system, $m,n,o,$ and $t$ are the discretizations of the Y-, X-, Z-, and t-coordinates, respectively. Let us generalize the numerical first derivative from WaveEquation.pdf (using the notation and names defined there) to numerical partial derivatives. We define nD such that, for a function $g(t,X,Y,Z)$,
\begin{align} \label{nDX and nDY}
\begin{split}
&\mbox{nD}_X(g_{m,n,o,t})\equiv\frac{g_{m,n+1,o,t}-g_{m,n-1,o,t}}{2\Delta} \approx \frac{\partial g}{\partial X}, \\ 
&\mbox{nD}_Y(g_{m,n,o,t})\equiv\frac{g_{m+1,n,o,t}-g_{m-1,n,o,t}}{2\Delta} \approx \frac{\partial g}{\partial Y}, \\
&\text{and similarly for $Z$ (index $o$, step size $\Delta$) and $t$ (index $t$, step size $\Delta t$).}
\end{split}
\end{align}
Using Eq. \ref{T}, we know that the left hand side of our main result, Eq. \ref{main eqn}, is given by
\begin{align} \label{expanded LHS}
LHS&=\frac{\partial}{\partial X}\Big[\frac{1}{8} (4 f \cos (2 \theta )+f \cos (4 \theta )+3 f+8) \frac{\partial u}{\partial X}+\frac{1}{8} f (6 \sin (2\theta )+\sin (4 \theta )) \frac{\partial u}{\partial Y}+f \sin ^2(\theta ) \cos (\theta ) \frac{\partial u}{\partial Z}\Big] \nonumber \\
&+\frac{\partial}{\partial Y}\Big[ \frac{1}{8} f (6 \sin (2 \theta )+\sin (4 \theta )) \frac{\partial u}{\partial X}+\frac{1}{8} (-8 f \cos (2 \theta) -f \cos (4 \theta )+f+8) \frac{\partial u}{\partial Y}-f \sin (\theta ) \cos ^2(\theta ) \frac{\partial u}{\partial Z}\Big] \nonumber \\
&+\frac{\partial}{\partial Z}\Big[ f \sin ^2(\theta ) \cos (\theta ) \frac{\partial u}{\partial X} -f \sin (\theta ) \cos ^2(\theta ) \frac{\partial u}{\partial Y}+\frac{1}{2} (f \cos (2 \theta )-f+2)\Big].
\end{align}
Now, we discretize Eq. \ref{expanded LHS} in terms of our previously defined numerical derivatives. For later convenience, let us multiply Eqs. \ref{nDX and nDY} by the spatial step size $\Delta$ and multiply the discretized form of Eq. \ref{expanded LHS} by $\Delta^2$ so as to put $LHS$ in terms of two functions, $\Delta\cdot\mbox{nD}_X$ and $\Delta\cdot\mbox{nD}_Y$, independent of parameters. This gives us
\begin{align} \label{delsqrLHS}
\Delta^2 \cdot LHS_{m,n,o,t}=\Delta\cdot \text{nD}_X&\Big[\frac{1}{8} (4 f_{m,n,o,t} \cos (2 \theta )+f_{m,n,o,t} \cos (4 \theta )+3 f_{m,n,o,t}+8) \Delta\cdot\text{nD}_X(u_{m,n,o,t}) \nonumber \\
&+\frac{1}{8} f_{m,n,o,t} (6 \sin (2\theta )+\sin (4 \theta )) \Delta\cdot\text{nD}_Y(u_{m,n,o,t}) \nonumber \\
&+f_{m,n,o,t} \sin ^2(\theta ) \cos (\theta ) \Delta\cdot\text{nD}_Z(u_{m,n,o,t})\Big] \nonumber \\
+\Delta\cdot \text{nD}_Y&\Big[ \frac{1}{8} f_{m,n,o,t} (6 \sin (2 \theta )+\sin (4 \theta )) \Delta\cdot\text{nD}_X(u_{m,n,o,t}) \nonumber \\
&+\frac{1}{8} (-8 f_{m,n,o,t} \cos (2 \theta) -f_{m,n,o,t} \cos (4 \theta )+f_{m,n,o,t}+8) \Delta\cdot\text{nD}_Y(u_{m,n,o,t}) \nonumber \\
&-f_{m,n,o,t} \sin (\theta ) \cos ^2(\theta ) \Delta\cdot\text{nD}_Z(u_{m,n,o,t})\Big] \nonumber \\
+\Delta\cdot \text{nD}_Z &\Big[ f_{m,n,o,t} \sin ^2(\theta ) \cos (\theta ) \Delta\cdot\text{nD}_X(u_{m,n,o,t}) -f_{m,n,o,t} \sin (\theta ) \cos ^2(\theta ) \Delta\cdot\text{nD}_Y(u_{m,n,o,t}) \nonumber \\
&+\frac{1}{2} (f_{m,n,o,t} \cos (2 \theta )-f+2) \Delta\cdot\text{nD}_Z(u_{m,n,o,t}) \Big].
\end{align}
For some uses, an explicit expression without nested functions will be necessary. We achieve this by applying Eqs. \ref{nDX and nDY} distributively\footnote{It is easy to check that this useful trick is allowed because $\mbox{nD}(kg)=k\mbox{nD}(g)$ for scalar k as $\mbox{nD}$ is, in principle, a linear transformation.} throughout Eq.\ref{delsqrLHS}. Computing through and evaluating at time $t-1$ produces (written, for brevity, in terms of the components of $\textbf{T}(f(t,X,Y,Z))$, given by Eq. \ref{T})
\begin{align} \label{delsqrLHS explicit}
\Delta^2 \cdot LHS_{m,n,o,t-1} &= \frac{1}{4}\Big[{T_{0,0}}_{m,n,o,t-1}(u_{m,n+2,o,t-1}-2u_{m,n,o,t-1}+u_{m,n-2,o,t-1}) \nonumber \\
+&{T_{0,1}}_{m,n,o,t-1}(u_{m+1,n+1,o,t-1}-u_{m+1,n-1,o,t-1}-u_{m-1,n+1,o,t-1}+u_{m-1,n-1,o,t-1}) \nonumber \\
+&{T_{0,2}}_{m,n,o,t-1}(u_{m,n+1,o+1,t-1}-u_{m,n-1,o+1,t-1}-u_{m,n+1,o-1,t-1}+u_{m,n-1,o-1,t-1}) \nonumber \\
+&{T_{1,0}}_{m,n,o,t-1}(u_{m+1,n+1,o,t-1}-u_{m-1,n+1,o,t-1}-u_{m+1,n-1,o,t-1}+u_{m-1,n-1,o,t-1}) \nonumber \\
+&{T_{1,1}}_{m,n,o,t-1}(u_{m+2,n,o,t-1}-2u_{m,n,o,t-1}+u_{m-2,n,o,t-1}) \nonumber \\
+&{T_{1,2}}_{m,n,o,t-1}(u_{m+1,n,o+1,t-1}-u_{m-1,n,o+1,t-1}-u_{m+1,n,o-1,t-1}+u_{m-1,n,o-1,t-1}) \nonumber \\
+&{T_{2,0}}_{m,n,o,t-1}(u_{m,n+1,o+1,t-1}-u_{m,n+1,o-1,t-1}-u_{m,n-1,o+1,t-1}+u_{m,n-1,o-1,t-1}) \nonumber \\
+&{T_{2,1}}_{m,n,o,t-1}(u_{m+1,n,o+1,t-1}-u_{m+1,n,o-1,t-1}-u_{m-1,n,o+1,t-1}+u_{m-1,n,o-1,t-1}) \nonumber \\
+&{T_{2,2}}_{m,n,o,t-1}(u_{m,n,o+2,t-1}-2u_{m,n,o,t-1}+u_{m,n,o-2,t-1} )\Big] \nonumber \\
=\ \frac{1}{4}\Big[{T_{0,0}}&_{m,n,o,t-1}(u_{m,n+2,o,t-1}-2u_{m,n,o,t-1}+u_{m,n-2,o,t-1}) \nonumber \\
+{T_{1,1}}&_{m,n,o,t-1}(u_{m+2,n,o,t-1}-2u_{m,n,o,t-1}+u_{m-2,n,o,t-1}) \nonumber \\
+{T_{2,2}}&_{m,n,o,t-1}(u_{m,n,o+2,t-1}-2u_{m,n,o,t-1}+u_{m,n,o-2,t-1} )\Big] \nonumber \\
+\frac{1}{2}\Big[{T_{0,1}}&_{m,n,o,t-1}(u_{m+1,n+1,o,t-1}-u_{m+1,n-1,o,t-1}-u_{m-1,n+1,o,t-1}+u_{m-1,n-1,o,t-1}) \nonumber \\
+{T_{0,2}}&_{m,n,o,t-1}(u_{m,n+1,o+1,t-1}-u_{m,n-1,o+1,t-1}-u_{m,n+1,o-1,t-1}+u_{m,n-1,o-1,t-1}) \nonumber \\
+{T_{1,2}}&_{m,n,o,t-1}(u_{m+1,n,o+1,t-1}-u_{m-1,n,o+1,t-1}-u_{m+1,n,o-1,t-1}+u_{m-1,n,o-1,t-1})\Big],
\end{align}
since each t-slice of $\textbf{T}$ is symmetric.

The right hand side of Eq. \ref{main eqn} is easily discretized in time. Equating that to $LHS$ gives us the fully discretized form of Eq. \ref{main eqn},
\begin{equation} \label{discr1}
\frac{1}{\Delta^2}\ \Delta^2 \cdot LHS_{m,n,o,t-1} = \frac{1}{c^2}\ \mbox{nD}_t(\mbox{nD}_t(u_{m,n,o,t-1})).
\end{equation}
Finally, applying Eqs. \ref{nDX and nDY} to Eq. \ref{discr1} and collecting parameters on one side, we have our desired recursively-defined solution of the wave equation in the presence of a gravitational wave.
\begin{align} \label{discr main eqn}
\boxed{
u_{m,n,o,t}=\left(c\ \frac{\Delta t}{\Delta}\right)^2 \Delta^2 \cdot LHS_{m,n,o,t-1}+2u_{m,n,o,t-1}-u_{m,n,o,t-2}
} \\
\text{with $\Delta^2 \cdot LHS_{m,n,o,t-1}$ given by Eq. \ref{delsqrLHS explicit}.} \nonumber
\end{align}
The manipulation that allowed us to program the numerical partial derivative functions independently of parameters now gives rise to the accumulated parameter
\begin{equation} \label{K}
K\equiv\left(c\ \frac{\Delta t}{\Delta}\right)^2.
\end{equation}
This is the same $K=c^2/c_i^2$ that we discussed in detail in WaveEquation.pdf. Its constraint conditions apply here as well.
\end{document}
